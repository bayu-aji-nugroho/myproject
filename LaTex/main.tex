\documentclass[11pt]{article}
\usepackage{amssymb} 
\title{peluang}
\author{bayu aji nugroho}
\begin{document}
\maketitle
\begin{enumerate}
    \item \textbf{aturan pejumlahan dan perkalian}
    \begin{enumerate}
        \item aturan penjumlahan\\
        rumus: $$ n_1 + n_2 + \dots + n_k $$ 
        \item aturan perkalian\\
        rumus $$ n_1 \times n_2 \times \dots \times n_k $$
    \end{enumerate}
    \item \textbf{faktorial}
    \\ notasi faktorial adalah $n!$ dengan n adalah bilangan asli \\ faktorial didefinisikan sebagai berikut
    $$ n! = (n-1)\times (n-2) \times \dots \times 2 \times 1$$ \\ atau
    $$ n! = \prod_{i=1}^{n} i$$
    contoh
    \begin{enumerate}
        \item $5! = 5\times 4 \times 3 \times 2 \times 1$
        \item $3! = 3\times 2\times 1$
    \end{enumerate}
    \newpage
    \item \textbf{permutasi dan kombinasi}
    \begin{enumerate}
        \item permutasi\\
        misal ada angka 1,2,3,4 berapakah cara membuat 2 digit angka dari  angka angka itu jika angka tidak boleh sama?\\
        
        \begin{tabular}{|c|c|}
            \hline
            4&3  \\
            \hline
            \end{tabular}
            \\ \\
        di baris 1 kolom 1 ada 4 angka yang bisa di masukkan dan di kolom ke 2
        hanya 3 karena satu angka sudah ada di kolom pertama 
        $$1 = {(1,2),(1,3),(1,4)}$$
        $$2 = {(2,1),(2,3),(2,4)}$$
        $$3 = {(3,1),(3,2),(3,4)}$$
        $$4 = {(4,1),(4,2),(4,3)}$$

        jika kita lihat ini sama dengan konsep perkalian dimana di kasus ini adalah $4\times 3$
        jadi banyak nya cara untuk menyusun angka 2 digit dari 4 angka itu adalah 12 cara\\
        bagaimana cara menghubungkannya dengan faktorial?\\
        $$4\times3 = \frac{4\times 3 \times 2 \times 1}{2\times 1} = \frac{4!}{2!} = \frac{4!}{(4-2)!}$$
        jadi jika 4 adalah banyaknya angka atau n dan 2 adalah
        banyaknya kotak  maka 
        $$\frac{n!}{(n-k)!}$$
        ini lah yang kita sebut sebagai permutasi jadi permutasi k unsur dari n unsur adalah
        $$P(n,k) = P_k^n = nPk = \frac{n!}{(n-k)!}$$
        \item permutasi berulang\\
        kita ambil kasus tadi tetapi angka boleh diulang maka yang terjadi adalah
        $$1 = {(1,1)(1,2),(1,3),(1,4)}$$
        $$2 = {(2,1),(2,2),(2,3),(2,4)}$$
        $$3 = {(3,1),(3,2),(3,3),(3,4)}$$
        $$4 = {(4,1),(4,2),(4,3),(4,4)}$$
        \newpage
        ini sama aja dengan $4\times 4$ atau $4^2$ jadi rumus umum untuk permutasi berulang adalah
        $$P(n,k) = n^k$$
        \item permutasi dengan unsur yang sama \\
        jika ada n unsur dimana n unsur tiru berisi $n_1$ unsur yang sama d
        dan $n_2$ unsur lain yang sama juga dan seterusnya sampai k maka
        $$P(n:n_1,n_2,\dots,n_k) = \frac{n!}{n_1!\times n_2!\times \dots \times n_k!}$$
        \item permutasi siklis \\ 
        adalah permutasi yang disusun melingkar
        $$p_s (n) = (n-1)!$$
    \end{enumerate}
    \item \textbf{Combinasi}\\
    bisa dijabarkan tapi malas(: \\
    intinya jika kita merujuk masalah pertama maka urutan yang beda itu sama contoh
    12 = 21
    $$C(n,k) = \frac{n!}{(n-k)!\times k!}$$
    \item \textbf{peluang}
    \begin{enumerate}
        \item ruang sempel\\
        adalah banyak cara kejadian dilakukan notasi $n(S)$
        \item peluang
        $$P(A)= \frac{n(A)}{s(A)}$$
        \begin{enumerate}
            \item A = kejadian A 
            \item P(A) = peluang kejadian A terjadi
            \item n(A) = banyak cara agar peluang A terjadi
            \item n(S) = banyaknya kemungkinan yang terjadi
        \end{enumerate}
        \item kejadian majemuk
        adalah kejadian 2 himpunan yang beriris dan selalu berlaku
        $$P(A \cup B ) = P(a)+P(B) - P(A \cap B)$$
        \begin{enumerate}
            \item kejadian saling lepas\\
            adalah kejadian yang tidak bisa terjadi bersama sama atau 2 himpunan yang tidak teriris
            karna tidak teriris maka $P(A\cap B) = 0$ dan karna terpisah maka
            $$P(A \cup B) = P(A) + P(B)$$
        \end{enumerate}
        \item kejadian saling bebas
        \\ adalah kejadian yang tidak saling mempengaruhi
        $$P(A\cap B) = P(A)\times P(B)$$
        
    \end{enumerate}
\end{enumerate}
\end{document}